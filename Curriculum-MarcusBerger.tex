\documentclass[a4paper, oneside, final]{scrartcl}

\usepackage[utf8x]{inputenc}
\usepackage[brazil]{babel}
\usepackage{amssymb}

\usepackage{soul}
\usepackage{scrpage2}
\usepackage{titlesec}
\usepackage{marvosym}
\usepackage{tabularx}
\usepackage{textcomp}

\usepackage[hmargin=1.5cm,vmargin=2.0cm,noheadfoot]{geometry}

\titleformat{\section}{\large\scshape\raggedright}{}{0em}{}[\titlerule]
\pagestyle{scrheadings}
\renewcommand{\headfont}{\normalfont\rmfamily\scshape}

\cofoot{
\so{ {\Large\Letter} marcusc-berger@hotmail.com \ {\Large\Telefon} +55 (11) 98530-1705 }
}

\begin{document}

\begin{center}
\textsc{\Huge{\so{Marcus Vinicius C. Berger}}}\\ \ \\

%\section{Objetivo}

Estagiar na área de desenvolvimento de software.

%\section{Resumo}

%\begin{tabularx}{0.97\linewidth}{X}
%  Estudante de Ciência da Computação segundo ano. \\ \ \\
%  Conhecimento básico em HTML, CSS, C, Linux, Windows e GitHub.
%\end{tabularx}

\section{Formação Acadêmica}

\begin{tabularx}{0.97\linewidth}{p{2cm}X}
2014$-$Atual & Cursando a graduação de Ciência da Computação\\
            & Universidade Paulista, UNIP\\ \\

\end{tabularx}

\begin{tabularx}{0.97\linewidth}{p{2cm}X}
2013$-$2014 & Técnico em Informática\\
            & Colégio Tableau\\
	    & Incompleto\\ \\	
\end{tabularx}

\section{Formação Complementar}

\begin{tabularx}{0.97\linewidth}{p{2cm}X}
2015        & Introdução a Programação (60h)\\
            & Universidade de São Paulo $-$ USP\\ \\
\end{tabularx}

\begin{tabularx}{0.97\linewidth}{p{2cm}X}
2015        & Try Git\\
            & Code School\\ \\
\end{tabularx}

\begin{tabularx}{0.97\linewidth}{p{2cm}X}
2015        & HTML Básico (8h)\\
            & CIEE\\ \\
\end{tabularx}

\section{Atuação Profissional}

\begin{tabularx}{0.97\linewidth}{p{2cm}X}

Período     & 01/2015 --- 06/2015\\
Empresa     & Universidade Federal de São Paulo - UNIFESP\\
Cargo       & Estagiario\\
Atividades  & Analisar ,conferir e arquivar documentos diversos, Auxiliar na elaboracao de documentos e planilhas, Acompanhar a manutencao de sistemas ou programa, Auxiliar na manutencao e desenvolvimento de novas funcionalidades nos sistemas de informacao da Unifesp.\\
%            & \ \\
\end{tabularx}

\section{Conhecimentos Básicos}

\begin{tabularx}{0.97\linewidth}{p{2cm}X}
        & HTML, CSS, C, Github, Linux, Eclipse e Windows\\
\end{tabularx}

\end{document}

